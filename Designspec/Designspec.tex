%Laborationsrapport

\documentclass[a4paper,12pt,fleqn]{article}
\usepackage{fixltx2e}
\usepackage[utf8]{inputenc}
\usepackage{graphicx}
\usepackage{sidecap}
\usepackage{fancyhdr}
\usepackage{amssymb,amsmath}
\usepackage[swedish]{babel}
\usepackage[margin=1.5in]{geometry}
\usepackage{abstract}
\usepackage[parfill]{parskip}
\usepackage{tocloft}
\usepackage{adjustbox}
\usepackage{textcomp}
\usepackage[math]{kurier}
\usepackage[T1]{fontenc}
\usepackage{listings}
\usepackage{xcolor}

\addto\captionsswedish{
  \renewcommand{\contentsname}%
    {Innehållsförteckning}%
}
\definecolor{listinggray}{gray}{0.9}
\definecolor{lbcolor}{rgb}{0.9,0.9,0.9}
\lstset{
backgroundcolor=\color{lbcolor},
    tabsize=4,    
%   rulecolor=,
    language=[GNU]C++,
        basicstyle=\scriptsize,
        upquote=true,
        aboveskip={1.5\baselineskip},
        columns=fixed,
        showstringspaces=false,
        extendedchars=false,
        breaklines=true,
        prebreak = \raisebox{0ex}[0ex][0ex]{\ensuremath{\hookleftarrow}},
        frame=single,
        numbers=left,
        showtabs=false,
        showspaces=false,
        showstringspaces=false,
        identifierstyle=\ttfamily,
        keywordstyle=\color[rgb]{0,0,1},
        commentstyle=\color[rgb]{0.026,0.112,0.095},
        stringstyle=\color[rgb]{0.627,0.126,0.941},
        numberstyle=\color[rgb]{0.205, 0.142, 0.73},
%        \lstdefinestyle{C++}{language=C++,style=numbers}’.
}
\lstset{
    backgroundcolor=\color{lbcolor},
    tabsize=4,
  language=C++,
  captionpos=b,
  tabsize=3,
  frame=lines,
  numbers=left,
  numberstyle=\tiny,
  numbersep=5pt,
  breaklines=true,
  showstringspaces=false,
  basicstyle=\footnotesize,
%  identifierstyle=\color{magenta},
  keywordstyle=\color[rgb]{0,0,1},
  commentstyle=\color{Darkgreen},
  stringstyle=\color{red}
  }

\renewcommand{\abstractnamefont}{\normalfont\normalsize\bfseries}
\renewcommand{\abstracttextfont}{\normalfont\small}
\renewcommand{\headrulewidth}{0pt}
\renewcommand{\cftsecleader}{\cftdotfill{\cftdotsep}} 
\setlength{\absleftindent}{0pt}
\setlength{\absrightindent}{0pt}
\setlength{\headheight}{15pt}

\addtolength{\oddsidemargin}{-.5in}
	\addtolength{\evensidemargin}{-.5in}
	\addtolength{\textwidth}{1in}




\pagestyle{fancy}
\lhead{
	\begin{picture}(0,0)
		\put(5,0){\includegraphics{logotyp.png}}
	\end{picture}}
	
\fancyhead[C]{\small{Mapmaster2001}}
\fancyhead[R]{\small{2014-02-xx}}
\fancyfoot[L]{\small{TSEA56 \\ LIPS Designspecifikation}}
\fancyfoot[C]{\small{\thepage}}
\fancyfoot[R]{\small{Projektgrupp 8 \\ Email}}

\renewcommand{\contentsname}{Innehållsförteckning}



\begin{document}
	\pagestyle{fancy}
\pagenumbering{roman}
	\vspace*{\fill}
		\begingroup
			\begin{center}
				\huge{\textbf{Designspecifikation}}
				\\
				\normalsize
				Kandidatprojekt Y - Grupp 8 - VT2014
				\\
				Version 0.1
				\end{center}
		\endgroup
	\vspace*{\fill}
	
	
	\begin{center} %Börjar centrering 
		Status
		\\
		\vspace{3pt} %Whitespace 3 pts
	    \begin{tabular}{| p{3cm} | p{3cm} | p{3cm} |} %tabell, 4 horizontella |, 3 cm emellan dem.
	    \hline %översta horizontella linjen.
	    Granskad & Av vem & Datum \\ \hline % & -tecken för att "gå till nästa ruta" 
		Godkänd & Av vem & Datum \\ \hline % avslutas med \\ och \hline.

	    \end{tabular}
	\end{center}
	\vspace{2cm}
	\newpage
	
	\vspace*{\fill}
		\begingroup
			\begin{center}
				\LARGE{\textbf{PROJEKTIDENTITET}}
				\\
				\footnotesize
				Grupp 8, 2014/VT, MapMaster2001
				\\
				Linköpings tekniska högskola, ISY
				\\
				\vspace{1cm}
	  \begin{tabular}{| p{3cm} | p{4.3cm} | p{2.4cm} | p{3.8cm} |}
	    \hline
		\textbf{Namn} & \textbf{Ansvar} & \textbf{Telefon} & \textbf{E-post} \\ \hline
	    Jens Edhammer & Dokumentanvsvarig (DOK) & 076-030 67 80 & jened502@student.liu.se \\ \hline
		Erik Ekelund & Designansvarig (DES) & 073-682 43 06 & eriek984@student.liu.se \\ \hline
		David Habrman &  & 976-017 71 15 & davha227@student.liu.se \\ \hline 
		Tobias Grundström & Testansvarig (TES) & 073-830 44 45 & tobgr602@student.liu.se \\ \hline 
		Hans-Filip Elo &   & 073-385 22 32 & hanel742@student.liu.se \\ \hline 
		Niklas Ericson & Projektledare (PL) & 073-052 27 05 & niker917@student.liu.se \\ \hline
	    \end{tabular}
		
		\vspace{1cm}
		\textbf{E-postlista för hela gruppen:} mapmaster2001@cyd.liu.se
		\\[0.5cm]
		
		\textbf{Kund}: Mattias Krysander, Linköpings Universitet, 581 83  LINKÖPING, \\
		013-28 21 98, matkr@isy.liu.se \\
		\textbf{Kontaktperson hos kund}: Mattias Krysander, 013-28 21 98,matkr@isy.liu.se 
		\\
		\textbf{Kursansvarig}: Tomas Svensson, 3B:528,013 28 21 59,tomass@isy.liu.se
		\\[0.5cm]
		\textbf{Handledare}: Peter Johansson, 013-28 1345 peter.a.johansson@liu.se

		

				\end{center}
		\endgroup
	\vspace*{\fill}
\newpage


	
\tableofcontents
\thispagestyle{fancy}
\newpage

\pagenumbering{arabic}
\section{Saker som kan vara bra att ha!}

\subsection{Punktlista}
\begin{itemize}
  \item Detta är en punktlista! 
  \item Punktlistor kan innehålla matte. $T_R \leq 10\%$
\end{itemize}

\subsection{Ekvationer}

som vi ser i ekvation~\ref{eq:test} %såhär refererar man till ekvationer och bilder.

\begin{equation}
F(s)= K\frac{\tau_D+1}{\beta\tau+1}\frac{\tau_I+1}{\tau_I+\gamma}	
	\label{eq:test} % glöm inte att ge eran ekvation en label för att kunna referera till den
\end{equation} 

\subsection{Bilder}


\begin{figure}[htp] %Placera här om det finns plats, annars så snart som möjligt, på toppen av en sida.
  \begin{center}
  \includegraphics[keepaspectratio=true,scale=0.8]{logotyp}  %skala och filnamn. 
  \end{center}
  \caption{Våran logotyp} %figurtext.
\end{figure}

\subsection{Kod}
Ibland vill man skriva kod, då kan man skriva i annan font. Förslagsvis:
\\ % <--- detta betyder ny rad btw.  


\section{Kommunikationsmodul}
Modul för att hantera kommunikation mellan robotens olika delkomponenter samt med persondatorn via Blåtand. Kommunikationsmodulen som syns i figur 3, kommer att agera master på robotens interna buss. Vid kommunikation mellan övriga moduler, dvs. sensor- och styrmodulen, kommer denna gå via kommunikationsmodulen.
Kommunikationsmodulen kommer alltså att leverera sensordata mellan roboten och mjukvaran som körs på persondatorn.

\subsection{Kommunikationsfall}
Ett par exempel på kommunikationsfall. \\

Kommunikationsmodulen skickar data mellan de olika enheterna. Ett par olika kommunikationsfall demonstreras nedan i flödesdiagrammet i figur~\ref{fig:flowcase1}. \\

Fall 1: Kommunikationsmodulen skickar manuella styrkommandon till styrmodulen\\
Fall 2: Sensormodulen signalerar att ny sensordata är redo.\\
Fall 3: Kartdata skickas från kommunikationsmodulen till PC. \\
Fall 4: PC signalerar nödstop \\

\begin{figure}[htp] %Placera här om det finns plats, annars så snart som möjligt, på toppen av en sida.
  \begin{center}
  \includegraphics[keepaspectratio=true,scale=0.8]{logotyp}  %skala och filnamn. 
  \end{center}
  \caption{Våran logotyp} %figurtext.
\end{figure}
\begin {lstlisting}

for (int i=0; i<iterations;i++)
{
do something hej hej
}
\end{lstlisting}



\end{document}