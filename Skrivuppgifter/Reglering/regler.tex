\documentclass[a4paper,12pt,fleqn]{article}
\usepackage{fixltx2e}
\usepackage[utf8]{inputenc}
\usepackage{graphicx}
\usepackage{sidecap}
\usepackage{fancyhdr}
\usepackage{amssymb,mathtools}
\usepackage[swedish]{babel}
\usepackage[margin=1.5in]{geometry}
\usepackage{abstract}
\usepackage[parfill]{parskip}
\usepackage{tocloft}
\usepackage{adjustbox}
\usepackage{textcomp}
\usepackage[T1]{fontenc}
\usepackage{listings}
\usepackage{xcolor,colortbl}
\usepackage{hyperref}

%----------------------------------------------------------------
%C-kod formatering

\definecolor{listinggray}{gray}{0.9}
\definecolor{lbcolor}{rgb}{0.9,0.9,0.9}
\lstset{
backgroundcolor=\color{lbcolor},
    tabsize=4,    
%   rulecolor=,
    language=[GNU]C++,
        basicstyle=\scriptsize,
        upquote=true,
        aboveskip={1.5\baselineskip},
        columns=fixed,
        showstringspaces=false,
        extendedchars=false,
        breaklines=true,
        prebreak = \raisebox{0ex}[0ex][0ex]{\ensuremath{\hookleftarrow}},
        frame=single,
        numbers=left,
        showtabs=false,
        showspaces=false,
        showstringspaces=false,
        identifierstyle=\ttfamily,
        keywordstyle=\color[rgb]{0,0,1},
        commentstyle=\color[rgb]{0.026,0.112,0.095},
        stringstyle=\color[rgb]{0.627,0.126,0.941},
        numberstyle=\color[rgb]{0.205, 0.142, 0.73},
%        \lstdefinestyle{C++}{language=C++,style=numbers}’.
}
\lstset{
    backgroundcolor=\color{lbcolor},
    tabsize=4,
  language=C++,
  captionpos=b,
  tabsize=3,
  frame=lines,
  numbers=left,
  numberstyle=\tiny,
  numbersep=5pt,
  breaklines=true,
  showstringspaces=false,
  basicstyle=\footnotesize,
%  identifierstyle=\color{magenta},
  keywordstyle=\color[rgb]{0,0,1},
  commentstyle=\color{Darkgreen},
  stringstyle=\color{red}
  }
  %-----------------------------------------------------------------
  %marginaler

  \renewcommand{\abstractnamefont}{\normalfont\normalsize\bfseries}
  \renewcommand{\abstracttextfont}{\normalfont\small}
  \renewcommand{\headrulewidth}{0pt}
  \renewcommand{\cftsecleader}{\cftdotfill{\cftdotsep}} 
  \setlength{\absleftindent}{0pt}
  \setlength{\absrightindent}{0pt}
  \setlength{\headheight}{15pt}

  \addtolength{\oddsidemargin}{-.5in}
  	\addtolength{\evensidemargin}{-.5in}
  	\addtolength{\textwidth}{1in}


  %-----------------------------------------------------------------
  %header and footer

  \pagestyle{fancy}
  \lhead{
  	\begin{picture}(0,0)
  		\put(5,0){\includegraphics{logotyp.png}}
  	\end{picture}}
	
  \fancyhead[C]{\small{Mapmaster2001}}
  \fancyhead[R]{\small \today}
  \fancyfoot[L]{\small{TSEA56 \\ LIPS Designspecifikation}}
  \fancyfoot[C]{\small{\thepage}}
  \fancyfoot[R]{\small{Projektgrupp 8 \\ Email}}

  %-----------------------------------------------------------------

%-------------------------------------------------------------------
%Första sidan

\begin{document}
	\pagestyle{fancy}
\pagenumbering{roman}
	\vspace*{\fill}
		\begingroup
			\begin{center}
				\huge{\textbf{Simultan positionering och kartläggning}}
				\\
				\vspace{10pt}
				\normalsize
				Tobias Grundström och Hans-Filip Elo
				\\
				Kandidatprojekt Y - Grupp 8 - VT2014
				\\
				Version 0.1
				\end{center}
		\endgroup
	\vspace*{\fill}

	\begin{center} %Börjar centrering 
		Status
		\\
		\vspace{3pt} %Whitespace 3 pts
	    \begin{tabular}{| p{3cm} | p{3cm} | p{3cm} |} %tabell, 4 horizontella |, 3 cm emellan dem.
	    \hline %översta horizontella linjen.
	    Granskad & - & \today \\ \hline % & -tecken för att "gå till nästa ruta" 
		Godkänd & - & - \\ \hline % avslutas med \\ och \hline.

	    \end{tabular}
	\end{center}
	\vspace{2cm}
	\newpage
%-----------------------------------------------------------------
%Projektidentitet

	\vspace*{\fill}
		\begingroup
			\begin{center}
				\LARGE{\textbf{PROJEKTIDENTITET}}
				\\
				\footnotesize
				Grupp 8, 2014/VT, MapMaster2001
				\\
				Linköpings tekniska högskola, ISY
				\\
				\vspace{1cm}
	  \begin{tabular}{| p{3cm} | p{4.3cm} | p{2.4cm} | p{3.8cm} |}
	    \hline
		\textbf{Namn} & \textbf{Ansvar} & \textbf{Telefon} & \textbf{E-post} \\ \hline
	    Jens Edhammer & Dokumentanvsvarig (DOK) & 076-030 67 80 & jened502@student.liu.se \\ \hline
		Erik Ekelund & Designansvarig (DES) & 073-682 43 06 & eriek984@student.liu.se \\ \hline
		David Habrman &  & 976-017 71 15 & davha227@student.liu.se \\ \hline 
		Tobias Grundström & Testansvarig (TES) & 073-830 44 45 & tobgr602@student.liu.se \\ \hline 
		Hans-Filip Elo &   & 073-385 22 32 & hanel742@student.liu.se \\ \hline 
		Niklas Ericson & Projektledare (PL) & 073-052 27 05 & niker917@student.liu.se \\ \hline
	    \end{tabular}

		\vspace{1cm}
		\textbf{E-postlista för hela gruppen:} mapmaster2001@cyd.liu.se
		\\[0.5cm]

		\textbf{Kund}: Mattias Krysander, Linköpings Universitet, 581 83  LINKÖPING, \\
		013-28 21 98, matkr@isy.liu.se \\
		\textbf{Kontaktperson hos kund}: Mattias Krysander, 013-28 21 98,matkr@isy.liu.se 
		\\
		\textbf{Kursansvarig}: Tomas Svensson, 3B:528,013 28 21 59,tomass@isy.liu.se
		\\[0.5cm]
		\textbf{Handledare}: Peter Johansson, 013-28 1345 peter.a.johansson@liu.se

				\end{center}
		\endgroup
	\vspace*{\fill}
\newpage

%-----------------------------------------------------------------
%Innehållsföreteckning

\addto\captionsswedish{\renewcommand{\contentsname}{Innehållsförteckning}}

\tableofcontents
\thispagestyle{fancy}
\newpage

\pagenumbering{arabic}
%-----------------------------------------------------------------
%Översikt

%------------------------------------------------
%--------------------Inledning-------------------
%------------------------------------------------
\section{Inledning}

Simultan positionering och kartläggning (SLAM) är ett problem som kan liknas vid ''Hönan och ägget''-problemet. Utan att veta var vi är - hur kartlägger vi då vår omgivining? Åt andra hållet får man ställa sig frågan - hur kartlägger vi vår omgivning utan att veta var vi är? 

Det är inte helt enkelt att lösa dessa frågor, men det finns approximativa lösningar på problemet SLAM. Gemensamt för alla lösningar är att de bygger på möjligheten att läsa av sin omgivning i kombination med
sannolikhetsteori. Då sensordata aldrig kan antas vara exakt använder man sannolikhetsteori för att göra rimlighetsbedömningar i de stickprov av sensordata som sensorerna ger oss.

Själva problemet är alltså inte entydigt löst rent matematiskt, utan
bygger på sannolikhetsteori i kombination med att moderna processorer
och minnen kan hantera en stor mängd data. Moderna processorer möjliggör
alltså ett stort stickprov vilket leder till en liten osäkerhet.

\subsection{Syfte}
Syftet med denna rapport är att fördjupa läsaren i de algoritmer och tekniker som används för kartläggning och positionsbestämning hos ett icke-medvetande system och att lösa ett enklare SLAM-problem.

\subsection{Historia}

Principerna för SLAM formulerades för första gången 1986\footnote{Smith, R.C.; Cheeseman, P}. Redan vid formuleringen av problemet menar man på att SLAM är en inexakt vetenskap. SLAM handlar om att skaffa sig en approximativ uppfattning av sin omgivning och position som är tillräckligt bra för att fatta ett beslut kring färdväg och/eller kartläggning. 

Utvecklingen på området har sedan dess accelererats kraftigt tack vare
mikrokontrollers förmåga att hantera mer data - då
sannolikhetsberäkningarna SLAM innefattar gynnas kraftigt av att arbeta
med stora stickprov.

På senare år har man, precis som inom många andra vetenskapliga områden, sett utvecklingen ta ytterligare ett kliv tack vare internet och öppen källkodsprojekt som Github och OpenSLAM. Att göra en sökning på ''SLAM'' på Github resulterar i en mängd aktiva projekt på området. Eftersom att källkoden där också finns tillgänglig är detta ett utmärkt exempel för de som vill lära sig om SLAM. 

%------------------------------------------------
%-------------Problemformulering-----------------
%------------------------------------------------

\section{Problemformulering}

Det problem som ska lösas är uppbyggt på följande sätt:

En robot med tillgång till ett antal avståndssensorer, ett gyro och en RFID-sensor placeras i ett stängt område med dimension 6x6 meter och som är uppdelad i 40x40 centimeters segment. Den ska, med en vägg som startpunkt, helt autonomt kartlägga området genom att markera ut var väggar finns och var det finns områden som inte går att nå. Den ska även markera ut var det går att finna RFID-taggar. Om inte hela det slutna området är kartlagt ska den upptäcka vilka segment som är oupptäckta, färdas dit och lägga till det i kartan. När området är kartlagt ska roboten ta sig tillbaka till startpunkten och avsluta arbetet.

%------------------------------------------------
%--------------------Kunskapsbas-----------------
%------------------------------------------------
\section{Kunskapsbas}

%----------------------Odometri------------------
\subsection{Odometri}

För att kunna utföra SLAM krävs det att man gör odometri, d v s kontinuerligt uppskattar vägen man färdats. Odometri kan göras på olika sätt - exempelvis genom att optiskt mäta avstånd till objekt i sin omgivning, vinkelhastigheter på hjul med känd storlek eller steg med given längd. 

Uppskattandet av färdvägen är aldrig en exakt lära. Det finns alltid en viss osäkerhet i sensorer. Det är av denna anledning som SLAM är en sannolikhetslära mer än en exakt vetenskap. 

De allra flesta robotar nyttjar förmågan att optiskt mäta objekt i sin omgivning. Anledningen till detta är att man då kan kompensera för tidigare mätfel vid framtida mätningar. Om man enbart förlitar sig på interna mätningar finns risken att tidiga mätfel görs, vilket fortplantar sig och leder till än större felskattningar senare under kartläggningen. 

Med optisk avläsning av omvärlden finns alltid möjligheten att korrigera tidigare felaktiga mätvärden genom att samla in mer mätdata för att mer korrekt kunna beskriva sin omgivning. Av den anledningen är all typ av SLAM beroende av att på något sätt granska sin omgivning. 

\subsubsection{Tillståndsrepresentation av reglersystem}

Inom reglerteknik kan linjära system beskrivas på så kallad tillståndsformel så som Glad, Torkel och Ljung, Lennart (2006) beskriver i \textit{Reglerteknik - Grundläggande teori}. I fallet med SLAM använder man lämpligen sensorernas värden som tillstånd, tecknat $\vec{x}$. Tillstånden mäts med diskreta värden då datorer endast hanterar diskreta mätningar och ej kontinuerliga. En diskret tillståndsbeskrivning av ett linjärt reglersystem beskrivs av: 

\begin{gather}
\vec{x}[n+1] = A\vec{x}[n] + B\vec{u}[n] \\
\vec{y}[n] = C\vec{x[n]} + D\vec{u[n]}
\end{gather}
\\
Där A, B, C och D är insignaler, u är insignaler och $\vec{y}$ exempelvis är kringliggande objekts position i förhållande till till roboten, alternativt robotens position. 

I ett realiserbart system nyttjas sensorer med en viss osäkerhet. Man kan därmed enbart skatta systemets tillstånd och inte exakt beräkna dem. 

Tillståndsvariablerna kan skattas med flera metoder utifrån tillgänglig mätdata. Två exempel på metoder Maximum likelihood-metoden (ML-skattning) och Minstakvadratmetoden. 

\paragraph{ML-skattning}

Antag att en sensors (stokastisk variabel) riktiga värde ges av $x_true$, och att $\hat{x} $ är ett givet stickprov (mätdata) med N antal mätningar. Alla tillstånd för $x_true$ antas vara lika sannolika, sensorvärdena är alltså binomialfördelade. 

Sannolikheten att $x = x_true$ ges då av: 

\begin{gather}
P(x_{true}) = 
\end{gather}
\\

\paragraph{Minstakvadratmetoden}

\subsubsection{Filtrering av sensordata}

Inom odometri finns det några olika tekniker för att filtrera sensordata från sin omgivning och besluta om var inom given inertialram sin nuvarande position är. Gemensamt för samtliga tekniker är att de är filter som arbetar mot oberoende tillståndsvariabler. 





%-------------------------------Typer av SLAM--------------------------------------------------------------------------------
\subsection{Typer av SLAM}
Det finns olika typer av SLAM för olika typer av tillämpningar och med olika egenskaper. Vissa är bra på större avstånd medan andra är mer precisa och bra på kortare avstånd. 

\subsubsection{Kalmanfilter (KF-SLAM)}
Ett Kalmanfilter är en typ av algoritm som med hjälp av tidigare mätdata från sensorer som kan innehålla brus och störningar uppskattar okända variabler. Ett Kalmanfilter arbetar på linjära och oberoende tillståndsvariabler och är ett essentiellt verktyg för vägval samt positionsuppskattningen i ett SLAM-system. 

Kalmnafilter är en lösning

\subsubsection{EKF-SLAM} 
EKF-SLAM är en förkortning för Extended Kalman Filter for Simultaneous Localization And Mapping. Det utökade Kalmanfiltret (extended) skiljer sig från det ursprungliga på så vis att den används vid icke-linjära förhållanden och istället linjäriserar variablerna kring ett medelvärde och deras kovarians. Med kovarians menas då hur de olika variablerna beror av varandra.

Algoritmen arbetar oftast i två steg; först skattas variablerna sedan vid nästa mätning 
kommer de skattade variablerna viktas, där de med mest säkerhet är av störst vikt.

\subsubsection{fastSLAM}
fastSlam är en teknik där systemet visuellt kan uppfatta landmärken. Landmärkenas position i förhållande till roboten noteras och uppmäts med regelbundna samplingar. Genom att positionsbestämma dessa landmärken kan roboten snabbare bestämma sin position. 

fastSLAM är snabbare på att bestämma sin position än KF-SLAM men är däremot inte genomförbar då miljön systemet rör sig i är så pass homogen att inga landmärken går att utfinna. Om vi tar den bana vår robot ska kartlägga som referens är den väldigt homogen\footnote{Tävlingsregler: \url{https://drive.google.com/file/d/0B758zzcy4ZrTeG1wRTY4WG9lTDQ/edit?usp=sharing}}. fastSLAM är antagligen inte den bästa tekniken för positionsbestämning och kartläggning av den typen av ''rum''. 



\subsubsection{VSLAM}

VSLAM står för visuell SLAM, vilket betyder att lokaliseringen sker med hjälp av kameror. Kamerorna används för att finna landmärken i området som kartläggs och sedan använda dessa som referenser när roboten fortsätter att utforska för att t ex se hur långt och i vilken riktning förflyttning sker relativt landmärket. Visuell SLAM är en förutsättning för att utnyttja den tidigare nämna fastSLAM-tekniken. 

För att bedöma avstånd till ett objekt mäts vinklar mellan farkost och objektet vid olika tidpunkter. Utifrån farkostens hastighet kan mjukvaran sedan bestämma avståndet till objektet i fråga. 

VSLAM är i de allra flesta fall komplicerat att implementera då man behöver avancerade bildsökningsalgoritmer för att finna lämpliga landmärken. 

\subsubsection{TSLAM}
Tactile SLAM är en metod som provats för robotar vid kartläggning av ett mörklagt område. Metoden använder sig av känselsensorer för att hitta avgränsningar i området och med hjälp av detta rita upp kartan. Denna metod ger dock inte så bra resultat med de tekniker som existerar i dag.

\subsubsection{WiFi-SLAM}
Denna metod använder sig av styrkan på WiFi-signaler i närheten för att avgöra positionen. Detta är något som använts på t.ex. mobiltelefoner för att avgöra var personer befinner sig. 

För att WiFi-SLAM ska fungera krävs att accesspunkten har information om positionsdata för sig själv, alternativt att den anslutna enheten har information om var aktuellt WiFi är tillgängligt. I en mobiltelefon används den senare metoden då telefonen kan kontrollera var WiFi-nätverket finns med hjälp av GPS. 

\subsection{Implementeringar}

Det finns väldigt många exempel på implementeringar av SLAM. Robotdammsugare är ett bra exempel på en modern tillämpning av SLAM. Det är en liten enhet som kan använda sig av en modifierad version av VSLAM som kallas CV-SLAM. Detta står för Ceiling Visual SLAM, det vill säga att roboten har kameror som är riktade uppåt och använder landmärken i taket för att rita upp en karta över rummet den städar, så att den dammsuger alla platser i rummet istället för att städa samma punkt flera gånger. 

Det här är ett väldigt bra exempel på hur moderna, små mikrokontrollers kan möjliggöra SLAM att användas för att förenkla vår vardag.  

SLAM används, och har används under längre tid, till exempel också i rymdexpeditioner där robotar skickas upp i rymden för att upptäcka och kartlägga ställen som vi människor inte har möjlighet att besöka. 

%------------------------------------------------
%--------------------Fördjupning-----------------
%------------------------------------------------
\newpage
\section{Fördjupning och kodexempel}

%------------------------------------------------
%------------Resultat och slutsatser-------------
%------------------------------------------------

\section{Resultat och slutsatser}

Till slut kan vi konstatera att SLAM används i många tillämpningar man kanske inte tänker på. Man kan också konstatera att algoritmerna, filtren och mjukvaran som används för att implementera SLAM relativt komplicerade. Vi inser att vi gjort ett klokt beslut att fördjupa oss i just SLAM, då vi är väldigt beroende av detta i vårt projekt att konstruera en kartritande robot. 

% --------------- Källförteckning ---------------------
\newpage \section{Källförteckning} Smith, R.C.; Cheeseman, P. (1986).
''On the Representation and Estimation of Spatial Uncertainty''. The
International Journal of Robotics Research, 5(4), sida 56–68. Hämtad
\today:
\url{http://www.frc.ri.cmu.edu/~hpm/project.archive/reference.file/Smith
&Cheeseman.pdf}

Glad, Torkel och Ljung, Lennart. 2006. \textit{Reglerteknik - Grundläggande teori}. Upplaga 4:10. Lund. Studentlitteratur AB.

Risgaard, S; Blas, M.R (2005).
''SLAM for Dummies, A Tutorial Approach to Simultaneous Localization and Mapping''. 
Hämtad \today:
\url{http://ocw.mit.edu/courses/aeronautics-and-astronautics/16-412j-cognitive-robotics-spring-2005/projects/1aslam_blas_repo.pdf}

Karlsson, N.; Goncalves, L.; Munich, M.E.; Pirjanian, P.
''The vSLAM Algorithm for Navigation in Natural Environments''. Evolution Robotics, Inc. Hämtad \today:
\url{http://www.vision.caltech.edu/mariomu/research/papers/vSLAM-krs.pdf}

Fox, C.; Evans, M.; Pearson, M.; Prescott, T. (2012)
''Tactile SLAM with a biomimetic whiskered robot''. 2012 IEEE International Conference on Robotics and Automation. Hämtad \today.
\url{http://ieeexplore.ieee.org/stamp/stamp.jsp?tp=&arnumber=6224813}

FastSLAM: A Factored Solution to the Simultaneous
Localization and Mapping Problem, Stanford University. Hämtad \today.
\url{http://robots.stanford.edu/papers/montemerlo.fastslam-tr.pdf}

Openslam.org
\url{http://www.openslam.org/}

Kandidatprojekt Y: Elektronikprojekt, Tävlingsregler för katläggningsrobot. Hämtad \today.  \url{https://drive.google.com/file/d/0B758zzcy4ZrTeG1wRTY4WG9lTDQ/edit?usp=sharing}

% ----------------------------- Appendix
% -----------------------------------------
% 
\newpage \appendix \pagestyle{empty}
\newgeometry{left=2cm,right=2cm,bottom=2cm,top=2cm} \section{Appendix A}

\end{document}