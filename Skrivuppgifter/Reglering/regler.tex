\documentclass[a4paper,12pt,fleqn]{article}
\usepackage{fixltx2e}
\usepackage[utf8]{inputenc}
\usepackage{graphicx}
\usepackage{sidecap}
\usepackage{fancyhdr}
\usepackage{amssymb,mathtools}
\usepackage[swedish]{babel}
\usepackage[margin=1.5in]{geometry}
\usepackage{abstract}
\usepackage[parfill]{parskip}
\usepackage{tocloft}
\usepackage{adjustbox}
\usepackage{textcomp}
\usepackage[T1]{fontenc}
\usepackage{listings}
\usepackage{xcolor,colortbl}
\usepackage{hyperref}

%----------------------------------------------------------------
%C-kod formatering

\definecolor{listinggray}{gray}{0.9}
\definecolor{lbcolor}{rgb}{0.9,0.9,0.9}
\lstset{
backgroundcolor=\color{lbcolor},
    tabsize=4,    
%   rulecolor=,
    language=[GNU]C++,
        basicstyle=\scriptsize,
        upquote=true,
        aboveskip={1.5\baselineskip},
        columns=fixed,
        showstringspaces=false,
        extendedchars=false,
        breaklines=true,
        prebreak = \raisebox{0ex}[0ex][0ex]{\ensuremath{\hookleftarrow}},
        frame=single,
        numbers=left,
        showtabs=false,
        showspaces=false,
        showstringspaces=false,
        identifierstyle=\ttfamily,
        keywordstyle=\color[rgb]{0,0,1},
        commentstyle=\color[rgb]{0.026,0.112,0.095},
        stringstyle=\color[rgb]{0.627,0.126,0.941},
        numberstyle=\color[rgb]{0.205, 0.142, 0.73},
%        \lstdefinestyle{C++}{language=C++,style=numbers}’.
}
\lstset{
    backgroundcolor=\color{lbcolor},
    tabsize=4,
  language=C++,
  captionpos=b,
  tabsize=3,
  frame=lines,
  numbers=left,
  numberstyle=\tiny,
  numbersep=5pt,
  breaklines=true,
  showstringspaces=false,
  basicstyle=\footnotesize,
%  identifierstyle=\color{magenta},
  keywordstyle=\color[rgb]{0,0,1},
  commentstyle=\color{Darkgreen},
  stringstyle=\color{red}
  }
  %-----------------------------------------------------------------
  %marginaler

  \renewcommand{\abstractnamefont}{\normalfont\normalsize\bfseries}
  \renewcommand{\abstracttextfont}{\normalfont\small}
  \renewcommand{\headrulewidth}{0pt}
  \renewcommand{\cftsecleader}{\cftdotfill{\cftdotsep}} 
  \setlength{\absleftindent}{0pt}
  \setlength{\absrightindent}{0pt}
  \setlength{\headheight}{15pt}

  \addtolength{\oddsidemargin}{-.5in}
  	\addtolength{\evensidemargin}{-.5in}
  	\addtolength{\textwidth}{1in}


  %-----------------------------------------------------------------
  %header and footer

  \pagestyle{fancy}
  \lhead{
  	\begin{picture}(0,0)
  		\put(5,0){\includegraphics{logotyp.png}}
  	\end{picture}}
	
  \fancyhead[C]{\small{Mapmaster2001}}
  \fancyhead[R]{\small \today}
  \fancyfoot[L]{\small{TSEA56 \\ LIPS Designspecifikation}}
  \fancyfoot[C]{\small{\thepage}}
  \fancyfoot[R]{\small{Projektgrupp 8 \\ Email}}

  %-----------------------------------------------------------------

%-------------------------------------------------------------------
%Första sidan

\begin{document}
	\pagestyle{fancy}
	 \pagenumbering{gobble}
	\vspace*{\fill}
		\begingroup
			\begin{center}
				\huge{\textbf{Simultan positionering och kartläggning}}
				\\
				\vspace{10pt}
				\normalsize
				Tobias Grundström och Hans-Filip Elo
				\\
				Kandidatprojekt Y - Grupp 8 - VT2014
				\\
				Version 0.3
				\end{center}
		\endgroup
	\vspace*{\fill}

	\begin{center} %Börjar centrering 
		Status
		\\
		\vspace{3pt} %Whitespace 3 pts
	    \begin{tabular}{| p{3cm} | p{3cm} | p{3cm} |} %tabell, 4 horizontella |, 3 cm emellan dem.
	    \hline %översta horizontella linjen.
	    Granskad & hanel742 & \today \\ \hline % & -tecken för att "gå till nästa ruta" 
		Godkänd & - & - \\ \hline % avslutas med \\ och \hline.

	    \end{tabular}
	\end{center}
	\vspace{2cm}
	\newpage
%-----------------------------------------------------------------
%Projektidentitet


	\vspace*{\fill}
		\begingroup
			\begin{center}
				\pagenumbering{roman}
				\LARGE{\textbf{PROJEKTIDENTITET}}
				\\
				\footnotesize
				Grupp 8, 2014/VT, MapMaster2001
				\\
				Linköpings tekniska högskola, Institutionen för Systemteknik (ISY)
				\\
				\vspace{1cm}
	  \begin{tabular}{| p{3cm} | p{4.3cm} | p{2.4cm} | p{3.8cm} |}
	    \hline
		\textbf{Namn} & \textbf{Ansvar} & \textbf{Telefon} & \textbf{E-post} \\ \hline
	    Jens Edhammer & Dokumentanvsvarig (DOK) & 076-030 67 80 & jened502@student.liu.se \\ \hline
		Erik Ekelund & Designansvarig (DES) & 073-682 43 06 & eriek984@student.liu.se \\ \hline
		David Habrman &  & 976-017 71 15 & davha227@student.liu.se \\ \hline 
		Tobias Grundström & Testansvarig (TES) & 073-830 44 45 & tobgr602@student.liu.se \\ \hline 
		Hans-Filip Elo &   & 073-385 22 32 & hanel742@student.liu.se \\ \hline 
		Niklas Ericson & Projektledare (PL) & 073-052 27 05 & niker917@student.liu.se \\ \hline
	    \end{tabular}

		\vspace{1cm}
		\textbf{E-postlista för hela gruppen:} mapmaster2001@cyd.liu.se
		\\[0.5cm]

		\textbf{Kund}: Mattias Krysander, Linköpings universitet, 581 83  LINKÖPING, \\
		013-28 21 98, matkr@isy.liu.se \\
		\textbf{Kontaktperson hos kund}: Mattias Krysander, 013-28 21 98,matkr@isy.liu.se 
		\\
		\textbf{Kursansvarig}: Tomas Svensson, 3B:528,013 28 21 59,tomass@isy.liu.se
		\\[0.5cm]
		\textbf{Handledare}: Peter Johansson, 013-28 1345 peter.a.johansson@liu.se

				\end{center}
		\endgroup
	\vspace*{\fill}
\newpage

%-----------------------------------------------------------------
%Innehållsföreteckning

\addto\captionsswedish{\renewcommand{\contentsname}{Innehållsförteckning}}

\tableofcontents
\thispagestyle{fancy}
\newpage

\pagenumbering{arabic}
%-----------------------------------------------------------------
%Översikt

%------------------------------------------------
%--------------------Inledning-------------------
%------------------------------------------------
\section{Inledning}

Simultan positionering och kartläggning (SLAM) är ett problem som grundar sig i följande frågeställning: Utan att veta var vi är - hur kartlägger vi då vår omgivning? Åt andra hållet får man ställa sig frågan - hur vet vi var vi befinner oss utan en karta?

Det är inte helt enkelt att lösa dessa frågor, men det finns approximativa lösningar på SLAM-problemet . Gemensamt för alla lösningar är att de bygger på möjligheten att läsa av sin omgivning i kombination med
sannolikhetsteori. Då sensordata aldrig kan antas vara exakt använder man sannolikhetsteori för att göra rimlighetsbedömningar i de stickprov av mätningar sensorerna ger.

SLAM-problemet är alltså oftast inte entydigt lösbart rent matematiskt, utan
bygger på sannolikhetsteori i kombination med att moderna processorer
och minnen kan hantera en stor mängd data. Moderna processorer möjliggör
alltså ett stort stickprov vilket kan leda till en mindre osäkerhet.


\subsection{Syfte}
Syftet med denna rapport är att ge läsaren en introduktion till de algoritmer och tekniker som används för kartläggning och positionsbestämning i ett mikroprocessorsystem.

\subsection{Historia}

Principerna för SLAM formulerades för första gången 1986\footnote{Smith, R.C.; Cheeseman, P. (1986).
''On the Representation and Estimation of Spatial Uncertainty''. The
International Journal of Robotics Research, 5(4), sida 56–68. Hämtad 28 mars 2014}. Redan vid formuleringen av problemet beskrevs SLAM som en inexakt vetenskap. SLAM handlar om att skaffa sig en approximativ uppfattning av sin omgivning och position som är tillräckligt bra för att fatta ett beslut kring färdväg och/eller kartläggning. 

Utvecklingen på området har sedan dess accelererat kraftigt tack vare
mikrokontrollers förmåga att hantera mer data. Då sannolikhetsberäkningarna SLAM innefattar gynnas kraftigt av att arbeta med stora stickprov. 

På senare år har man, precis som inom många andra vetenskapliga områden, sett utvecklingen ta ytterligare ett kliv tack vare internet och öppna källkodsprojekt som Github och OpenSLAM. Att göra en sökning på SLAM på Github resulterar i en mängd aktiva projekt på området. Eftersom källkoden där också finns tillgänglig är detta ett utmärkt exempel för de som vill läsa på om SLAM-algoritmer. 

%------------------------------------------------
%-------------Problemformulering-----------------
%------------------------------------------------

\section{Problemformulering}

En robot med tillgång till ett antal avståndssensorer och ett gyro placeras i ett okänt slutet område. Den ska, med en vägg som startpunkt, helt autonomt kartlägga området genom att markera ut var väggar finns och var det finns områden som inte går att nå. Om inte hela det slutna området är kartlagt ska den upptäcka vilka segment som är oupptäckta, färdas dit och lägga till det i kartan. När området är kartlagt ska roboten ta sig tillbaka till startpunkten för att kunna återhämtas och på så sätt avsluta arbetet. 

SLAM är en teknik där ett system visuellt kan uppfatta landmärken. Med landmärken menas alla typer av objekt eller liknande som kan användas som referenser. Landmärkenas position i förhållande till roboten ska noteras och uppmätas med regelbundna samplingar. Genom att positionsbestämma dessa landmärken kan roboten snabbare bestämma sin position. 

Denna rapport kommer behandla en del av problemen som uppstår vid SLAM och vad som behövs för att lösa dem.


%------------------------------------------------
%--------------------Fördjupning-----------------
%------------------------------------------------
\newpage
\section{Simultaneous Localization And Mapping}
I denna fördjupande del går vi igenom förkunskaper som krävs för att beskriva en implementering som kan utföra SLAM i en miljö lik den i problemställningen. Bland annat kommer tillståndsrepresentation, skattning av tillstånd med hjälp av observatörer och Kalmanfilter tas upp.

För att kunna utföra SLAM krävs det att man gör odometri, det vill säga att man kontinuerligt uppskattar vägen som färdats. Odometri kan göras på olika sätt - exempelvis genom att optiskt mäta avstånd till objekt i sin omgivning, vinkelhastigheter på hjul med känd storlek eller steg med given längd. Många robotar nyttjar förmågan att optiskt mäta objekt i sin omgivning, men det förekommer även implementeringar med radar, sonar eller GPS. Uppskattandet av färdvägen är aldrig en exakt lära. Det finns alltid en viss osäkerhet i sensorer. Det är av denna anledning som SLAM är en sannolikhetslära mer än en exakt vetenskap. 

Med optisk avläsning av omvärlden finns alltid möjligheten att korrigera tidigare felaktiga mätvärden genom att samla in mer mätdata för att på ett korrektare sätt kunna beskriva sin omgivning. Av den anledningen är all typ av SLAM beroende av att på något sätt granska sin omgivning. 

\subsection{Mätningar}

För att kunna avgöra var roboten befinner sig behövs ett eller flera sätt för att känna av omgivningen. Den kan till exmepel behöva utföra mätningar på avstånd, hastighet och vinkelhastighet för att åstadkomma detta. Dessa mätningar och uppskattningar kan göras på många sätt, men görs mestadels med olika typer av sensorer så som avståndssensorer, accelerometrar och gyroskop i kombination med en dator som processerar datan.

\subsubsection{Avståndssensor}

Som avståndsmätare kan någon typ av optisk sensor användas. Att sensorn är optisk innebär att den använder sig av ljus för att utföra mätningarna. De finns till exempel sensorer som belyser en yta som avstånd ska mätas till, tar emot reflekterat ljus och sedan med hjälp av tiden det tagit för ljuset att färdas beräkna avståndet. Denna metod kallas time-of-flight. Ytterligare en metod som används kallas optisk triangulering och det innebär att med hjälp av vinkeln mellan sändare och mottagare bestämma avståndet till objektet\footnote{Hägg, M. Avståndskamera för utomhusbruk, LTU http://epubl.ltu.se/1402-1617/2000/317/LTU-EX-00317-SE.pdf, hämtad 2014-05-06}.

\subsubsection{Gyroskop}

För att mäta vinkelhastighet går det att använda ett gyroskop även kallat gyro. En gyrosensor av MEMS-typ (Micro Electro-Mechanical System) utnyttjar att två lika stora massor som oscillerar konstant i motsatta riktningar påverkas av Korioliskraften vid rotation. Kraften verkar i olika riktningar på de två massorna och det leder till en förändring i kapacitans.
Skillnaden i kapacitans har visat sig vara proportionell mot vinkelhastigheten, därav går det att bestämma den med hjälp av ett gyro \footnote{http://electroiq.com/blog/2010/11/introduction-to-mems-gyroscopes/, hämtad 2014-05-06}.

\subsubsection{Processering av data}

Sensorerna producerar mängder med data, data som kan variera något vid varje mätning. Om det då utgås från att närliggande mätningar görs kring samma position för roboten kan mätningarna antas vara normalfördelade.

\begin{figure}[htp] %Placera här om det finns plats, annars så snart som möjligt, på toppen av en sida.
  \begin{center}
  \includegraphics[keepaspectratio=true,scale=0.5]{normalfordelning.png}  %skala och filnamn. 
  \end{center}
  \caption{Normalfördelning} %figurtext.
  \label{fig:fire} %glöm inte att uppdatera era labels
\end{figure}

En mer korrekt bild av till exempel avståndet kan sedan fås med mätningarnas väntevärde, $\mu$. Då mätningarna är normalfördelade ges väntevärdet av mätningarnas medelvärde, $\bar{x}$, vilket kan ses i figur ovan\footnote{Wikipedia, \url{http://sv.wikipedia.org/wiki/Normalf\%C3\%B6rdelning}, hämtad 2014-05-02.}. 

\subsubsection{Andra typer av sensorer}

I denna rapport berör vi mest området där SLAM baseras på optiska avståndssensorer, men det finns även andra typer av sensorer som går att använda. Till exempel finns det implementeringar som använder sig av kameror för att finna landmärken och sedan använda dessa som referenser. Det är dock mer avancerade algoritmer som krävs för att finna lämpliga landmärken \footnote{Karlsson, N.; Goncalves, L.; Munich, M.E.; Pirjanian, P.''The vSLAM Algorithm for Navigation in Natural Environments''. Evolution Robotics, Inc. Hämtad 28 mars 2014}.

Något som också provats är SLAM utan användning av någon typ av optisk utrustning och istället förlita sig på känsel, med hjälp av känselsensorer som efterliknar djurs morrhår, för att kunna kartlägga ett helt mörklagt rum. Denna metod ger dock inte så bra resultat med de tekniker som existerar i dag\footnote{Fox, C.; Evans, M.; Pearson, M.; Prescott, T. (2012)
''Tactile SLAM with a biomimetic whiskered robot''. 2012 IEEE International Conference on Robotics and Automation. Hämtad 28 mars 2014.
\url{http://ieeexplore.ieee.org/stamp/stamp.jsp?tp=&arnumber=6224813}
}.

Tidigare nämndes till exempel metoder som använder sig av GPS. Detta i kombination med WiFi-signaler har använts till exempel i mobiltelefoner för att avgöra var personer befinner sig. För att WiFi-SLAM ska fungera krävs att accesspunkten har information om positionsdata för sig själv, alternativt att den anslutna enheten har information om var aktuellt WiFi är tillgängligt.


%----------------------Tillståndsrepresentation av reglersystem------------------

\subsection{Tillståndsrepresentation av reglersystem}

Inom reglerteknik kan linjära system beskrivas på så kallad tillståndsform så som Glad och Ljung\footnote{Glad, Torkel och Ljung, Lennart (2006), \textit{Reglerteknik - Grundläggande teori}.} beskriver. I fallet med SLAM för en robot använder man robotens position som tillstånd , tecknat $x$. Ett tidskontinuerligt system kan beskrivas på tillståndsform som nedan.
\begin{gather}
\dot{x}=Ax+Bu \\
y=Cx+Du	
\label{equ:tillstand}
\end{gather}
där $A$, $B$, $C$ och $D$ är matriser, $u$ är en vektor med insignaler och $y$ är en vektor med utsignaler.

Tillstånden för en robot mäts med tidsdiskreta värden då datorer endast hanterar diskreta mätningar och ej kontinuerliga. I dessa fall är det bättre att använda sig av en diskret tillståndsbeskrivning. En diskret tillståndsbeskrivning av ett linjärt reglersystem beskrivs av: 
\begin{gather}
x[n+1] = Ax[n] + Bu[n] \\
y[n] = Cx[n] + Du[n]
\end{gather}

Om systemet istället skulle vara icke-linjärt så går det inte att beskriva det på samma sätt. Det kan dock approximeras genom att linjärisera systemet genom att använda sig av Taylorutveckling, vilket i slutändan leder till att $A$-, $B$-, $C$- och $D$-matriserna innehåller partiella derivator. Dessa matriser kallas jacobianer, $J$ och skrivs:
\begin{gather}
	J= \begin{pmatrix}
	\frac{\partial f_1}{\partial x_1} & \dots & \frac{\partial f_n}{\partial x_1} \\
	  							\vdots &       & \vdots \\
	  \frac{\partial f_1}{\partial x_n} & \dots & \frac{\partial f_n}{\partial x_n}
	  \end{pmatrix}
\end{gather}



I ett realiserbart system nyttjas till exempel sensorer som insignaler med en viss osäkerhet för att bestämma systemets tillstånd. Man kan därmed enbart skatta systemets tillstånd och inte exakt beräkna dem. För att uppskatta dem används det ofta en så kallad observatör.

\subsubsection{Tillståndsåterkoppling och observatörer}

Inom reglerteknik används en observatör för att skatta tillståndsvariabler i ett givet system när tillstånden hos systemet ej kan ges med säkerhet. I fallet positionering och kartritning kan roboten och kringliggande objekts positioner som sagt inte bestämmas entydigt och behöver därför skattas. 

Skattningen av variabler kan utföras genom tillståndsåterkoppling, som sedan verifieras med observatören.  Tillståndet $x$ för systemet i ekvation \ref{equ:tillstand} kan skattas med hjälp av observatören: 

\begin{gather}
\dot{\hat{x}} = Ax + Bu + K(y - C\hat{x})
\label{equ:observer}
\end{gather}

Skattningsfelet ges sedan av differentialekvationen: 

\begin{gather}
\dot{\tilde{x}} = (A - KC)\tilde{x}
\label{equ:observerError}
\end{gather}

Genom att välja olika K-matriser ges systemet olika egenskaper i form av hastighet och störningskänslighet. Inom SLAM är både hastighet och störningskänslighet essentiellt och man använder då en variabel K-matris. För att skatta nästkommande tillstånd med variabel K-matris kan man använda ett system som kallas Kalmanfilter. 

\subsubsection{Kalmanfilter}

Antag att tillståndssystemet i ekvation \ref{equ:tillstand} störs av mätfelet $v$ och yttre störningen $e$, vilket vi antar är vitt brus.  
\begin{gather}
\dot{x}=Ax+Bu+e \\
y=Cx+Du+v
\end{gather}

Mätfelet i ekvation \ref{equ:observerError} ges då istället av: 
\begin{gather}
\dot{\tilde{x}} = (A - KC)\tilde{x} + e - Kv
\end{gather}

Då $e$ och $v$ antas vara vitt brus kan man, genom att uppskatta störningarnas kovarians, beräkna mätfelets kovarians. Kovariansen hos mätfelet minimeras sedan genom att sätta K enligt: 
\begin{gather}
K = PC^{T}R_{2}^{-1}
\end{gather}

Där matrisen $P$ ges av att lösa ekvationen nedan - och sedan välja den positivt semidefinita lösningen på $P$. 
\begin{gather}
AP + PA^{T} + R_{1} - PC^{T}R_{2}^{-1}CP = 0
\end{gather}

I praktiska tillämpningar antas $R_{1}$ och $R_{2}$ vara diagonalmatriser för att förenkla implementering. Dessa matriser viktar sedan hur mycket störning man uppskattar att systemet påverkas av. 

Kalmanfiltret är tyvärr inte särskilt effektivt vid SLAM. Detta då filtret enbart är implementerbart för linjära system. Av den anledningen används istället ett anpassat Kalmanfilter vid implementeringar av SLAM.  

\subsubsection{EKF - Det utökade kalmanfiltret}

Det utökade kalmanfiltret, Extended Kalman Filter (EKF), är vad som vanligtvis används då man implementerar SLAM-algoritmer. Till skillnad från Kalmanfiltret kan EKF lösa icke-linjära SLAM-problem. Detta då filtrets matriser beräknas om för varje nytt tillstånd. Matriserna beräknas med hjälp av jacobianer, och linjäriseras därmed kring varje tillstånd för att kunna skatta nästa tillstånd. 

\subsection{FastSLAM}
FastSLAM är en relativt modern teknik som använder ett så kallat partikelfilter för att filtrera givet mätdata, där varje värde i mätdatat ses som en partikel. De mest sannolika värdena sparas och de minst sannolika värdena förkastas. Man kallar det här för en omsapling av datat då man utgår från en sampelmängd av mätvärden och sedan minskar denna. 

För att effektivisera filtreringen av mätvärden delas olika distinkt upptäckta objekt i robotens miljö in i olika zoner. Dessa zoner filtreras sedan individuellt för att få fram den mest sannolika positionen för objektet i zonen. 

Mätvärdena körs sedan flera gånger genom filtret tills dess att man med tillräckligt stor sannolikhet kan bestämma kartans utseende. Metoden har visat sig väldigt effektiv i praktiken. Nackdelen med metoden är att man genom omsampling av mätvärdena förlorar mätinformation som skulle kunna vara korrekt. Beroende på hur filtret prioriterar är det möjligt att man får en skev bild av omgivningen.

De matematiska grunderna för FastSLAM ligger utanför vad denna rapport täcker in - men för den intresserade läsaren rekommenderas David Törnqvists doktorsavhandling \textit{Estimation and Detection with Applications to Navigation}\footnote{David Törnqvist. Estimation and Detection with Applications to Navigation. PhD thesis, Linköping University, 2008. Hämtad 15 april 2014.}.

\subsection{Exempel på andra implementeringar}

Det finns många exempel på implementeringar av SLAM. Robotdammsugare är ett bra exempel på en modern tillämpning som kan använda sig av SLAM. Det är då en liten enhet som kan utnyttja en modifierad version av VSLAM som kallas CV-SLAM. Detta står för Ceiling Visual SLAM, det vill säga att roboten har kameror som är riktade uppåt och använder landmärken i taket för att rita upp en karta över rummet den städar. Det gör att den dammsuger alla platser i rummet istället för att städa samma punkt flera gånger. 

Det här är ett bra exempel på hur moderna, små mikrokontrollers kan göra det möjligt att använda SLAM för att förenkla vår vardag.

SLAM används, och har använts under längre tid, också i rymdexpeditioner där robotar skickas upp i rymden för att upptäcka och kartlägga ställen som vi människor inte har möjlighet att besöka. 

%------------------------------------------------
%------------Resultat och slutsatser-------------
%------------------------------------------------
\newpage
\section{Resultat och slutsatser}

Till slut kan vi konstatera att SLAM används i många tillämpningar man kanske inte tänker på. Man kan också konstatera att algoritmerna, filtren och mjukvaran som används för att implementera SLAM ibland kan vara väldigt komplexa. I miljöer där sensorerna kan täcka en stor del av omgivningen kan däremot algoritmerna hållas relativt enkla. Robotens position i förhållande till kringliggande objekt kan lätt skattas med avståndssensorernas medelvärde. 

Genom att ta stora stickprov, alltså många samplingar, och sedan beräkna medelvärdet av dem filtreras eventuella avvikande mätvärden bort på ett enkelt sätt. 

Vi inser att vi gjort ett klokt beslut att fördjupa oss i just SLAM, då vi är beroende av detta i vårt projekt att konstruera en kartritande robot. SLAM var också den delen vi såg, och fortfarande ser, som störst hinder i utvecklingsprocessen. 

Vi har insett att vi tack vare den relativt långa räckvidden hos våra optiska sensorer i kombination med det relativt lilla område vi ska kartlägga kommer kunna använda en relativt enkel implementering. En implementering som använder sig av ML-skattning som observatör\footnote{Glad och Ljung (2006)} för systemet. 

% --------------- Källförteckning ---------------------
\newpage 
\section*{Källförteckning} 
\addcontentsline{toc}{section}{Källförteckning}

David Törnqvist. Estimation and Detection with Applications to Navigation. PhD thesis, Linköping University, 2008. Hämtad 15 april 2014.
\url{http://urn.kb.se/resolve?urn=urn:nbn:se:liu:diva-14956}

Datablad Atmega 1284p, Vanheden, databladsserver hos Institutionen för Systemteknik vid Linköpings Universitet. Hämtad 14 april 2014. \url{https://docs.isy.liu.se/twiki/pub/VanHeden/DataSheets/atmega1284p.pdf}

FastSLAM: A Factored Solution to the Simultaneous
Localization and Mapping Problem, Stanford University. Hämtad 28 mars 2014.
\url{http://robots.stanford.edu/papers/montemerlo.fastslam-tr.pdf}

Fox, C.; Evans, M.; Pearson, M.; Prescott, T. (2012)
''Tactile SLAM with a biomimetic whiskered robot''. 2012 IEEE International Conference on Robotics and Automation. Hämtad 28 mars 2014.
\url{http://ieeexplore.ieee.org/stamp/stamp.jsp?tp=&arnumber=6224813}

Glad, Torkel och Ljung, Lennart. 2006. \textit{Reglerteknik - Grundläggande teori}. Upplaga 4:10. Lund. Studentlitteratur AB.

Kandidatprojekt Y: Elektronikprojekt, Tävlingsregler för katläggningsrobot. Hämtad 28 mars 2014.  \url{https://drive.google.com/file/d/0B758zzcy4ZrTeG1wRTY4WG9lTDQ/edit?usp=sharing}

Karlsson, N.; Goncalves, L.; Munich, M.E.; Pirjanian, P.
''The vSLAM Algorithm for Navigation in Natural Environments''. Evolution Robotics, Inc. Hämtad 28 mars 2014:
\url{http://www.vision.caltech.edu/mariomu/research/papers/vSLAM-krs.pdf}

Openslam.org
\url{http://www.openslam.org/}

Risgaard, S; Blas, M.R (2005).
''SLAM for Dummies, A Tutorial Approach to Simultaneous Localization and Mapping''. 
Hämtad 28 mars 2014:
\url{http://ocw.mit.edu/courses/aeronautics-and-astronautics/16-412j-cognitive-robotics-spring-2005/projects/1aslam_blas_repo.pdf}

Smith, R.C.; Cheeseman, P. (1986). ''On the Representation and Estimation of Spatial Uncertainty''. The
International Journal of Robotics Research, 5(4), sida 56–68. Hämtad
28 mars 2014:
\url{http://www.frc.ri.cmu.edu/~hpm/project.archive/reference.file/Smith&Cheeseman.pdf}






\end{document}