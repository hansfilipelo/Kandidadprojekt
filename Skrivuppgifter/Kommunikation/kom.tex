%Laborationsrapport

\documentclass[a4paper,12pt,fleqn]{article}
\usepackage{fixltx2e}
\usepackage[utf8]{inputenc}
\usepackage{graphicx}
\usepackage{sidecap}
\usepackage{fancyhdr}
\usepackage{amssymb,amsmath}
\usepackage[swedish]{babel}
\usepackage[margin=1.5in]{geometry}
\usepackage{abstract}
\usepackage[parfill]{parskip}
\usepackage{tocloft}
\usepackage{adjustbox}
\usepackage{textcomp}
\usepackage[T1]{fontenc}
\usepackage{listings}
\usepackage{xcolor,colortbl}
\usepackage{hyperref}

%----------------------------------------------------------------
%C-kod formatering

\definecolor{listinggray}{gray}{0.9}
\definecolor{lbcolor}{rgb}{0.9,0.9,0.9}
\lstset{
backgroundcolor=\color{lbcolor},
    tabsize=4,    
%   rulecolor=,
    language=[GNU]C++,
        basicstyle=\scriptsize,
        upquote=true,
        aboveskip={1.5\baselineskip},
        columns=fixed,
        showstringspaces=false,
        extendedchars=false,
        breaklines=true,
        prebreak = \raisebox{0ex}[0ex][0ex]{\ensuremath{\hookleftarrow}},
        frame=single,
        numbers=left,
        showtabs=false,
        showspaces=false,
        showstringspaces=false,
        identifierstyle=\ttfamily,
        keywordstyle=\color[rgb]{0,0,1},
        commentstyle=\color[rgb]{0.026,0.112,0.095},
        stringstyle=\color[rgb]{0.627,0.126,0.941},
        numberstyle=\color[rgb]{0.205, 0.142, 0.73},
%        \lstdefinestyle{C++}{language=C++,style=numbers}’.
}
\lstset{
    backgroundcolor=\color{lbcolor},
    tabsize=4,
  language=C++,
  captionpos=b,
  tabsize=3,
  frame=lines,
  numbers=left,
  numberstyle=\tiny,
  numbersep=5pt,
  breaklines=true,
  showstringspaces=false,
  basicstyle=\footnotesize,
%  identifierstyle=\color{magenta},
  keywordstyle=\color[rgb]{0,0,1},
  commentstyle=\color{Darkgreen},
  stringstyle=\color{red}
  }
  %-----------------------------------------------------------------
  %marginaler

  \renewcommand{\abstractnamefont}{\normalfont\normalsize\bfseries}
  \renewcommand{\abstracttextfont}{\normalfont\small}
  \renewcommand{\headrulewidth}{0pt}
  \renewcommand{\cftsecleader}{\cftdotfill{\cftdotsep}} 
  \setlength{\absleftindent}{0pt}
  \setlength{\absrightindent}{0pt}
  \setlength{\headheight}{15pt}

  \addtolength{\oddsidemargin}{-.5in}
  	\addtolength{\evensidemargin}{-.5in}
  	\addtolength{\textwidth}{1in}


  %-----------------------------------------------------------------
  %header and footer

  \pagestyle{fancy}
  \lhead{
  	\begin{picture}(0,0)
  		\put(5,0){\includegraphics{logotyp.png}}
  	\end{picture}}
	
  \fancyhead[C]{\small{Mapmaster2001}}
  \fancyhead[R]{\small \today}
  \fancyfoot[L]{\small{TSEA56 \\ LIPS Designspecifikation}}
  \fancyfoot[C]{\small{\thepage}}
  \fancyfoot[R]{\small{Projektgrupp 8 \\ Email}}

  %-----------------------------------------------------------------

%-------------------------------------------------------------------
%Första sidan

\begin{document}
	\pagestyle{fancy}
\pagenumbering{roman}
	\vspace*{\fill}
		\begingroup
			\begin{center}
				\huge{\textbf{Trådlös kommunikation}}
				\\
				\vspace{10pt}
				\normalsize
				Niklas Ericson och Jens Edhammer
				\\
				Kandidatprojekt Y - Grupp 8 - VT2014
				\\
				Version 1.0
				\end{center}
		\endgroup
	\vspace*{\fill}
	
	\begin{center} %Börjar centrering 
		Status
		\\
		\vspace{3pt} %Whitespace 3 pts
	    \begin{tabular}{| p{3cm} | p{3cm} | p{3cm} |} %tabell, 4 horizontella |, 3 cm emellan dem.
	    \hline %översta horizontella linjen.
	    Granskad & - & \today \\ \hline % & -tecken för att "gå till nästa ruta" 
		Godkänd & - & - \\ \hline % avslutas med \\ och \hline.

	    \end{tabular}
	\end{center}
	\vspace{2cm}
	\newpage
%-----------------------------------------------------------------
%Projektidentitet
	
	\vspace*{\fill}
		\begingroup
			\begin{center}
				\LARGE{\textbf{PROJEKTIDENTITET}}
				\\
				\footnotesize
				Grupp 8, 2014/VT, MapMaster2001
				\\
				Linköpings tekniska högskola, ISY
				\\
				\vspace{1cm}
	  \begin{tabular}{| p{3cm} | p{4.3cm} | p{2.4cm} | p{3.8cm} |}
	    \hline
		\textbf{Namn} & \textbf{Ansvar} & \textbf{Telefon} & \textbf{E-post} \\ \hline
	    Jens Edhammer & Dokumentanvsvarig (DOK) & 076-030 67 80 & jened502@student.liu.se \\ \hline
		Erik Ekelund & Designansvarig (DES) & 073-682 43 06 & eriek984@student.liu.se \\ \hline
		David Habrman &  & 976-017 71 15 & davha227@student.liu.se \\ \hline 
		Tobias Grundström & Testansvarig (TES) & 073-830 44 45 & tobgr602@student.liu.se \\ \hline 
		Hans-Filip Elo &   & 073-385 22 32 & hanel742@student.liu.se \\ \hline 
		Niklas Ericson & Projektledare (PL) & 073-052 27 05 & niker917@student.liu.se \\ \hline
	    \end{tabular}
		
		\vspace{1cm}
		\textbf{E-postlista för hela gruppen:} mapmaster2001@cyd.liu.se
		\\[0.5cm]
		
		\textbf{Kund}: Mattias Krysander, Linköpings Universitet, 581 83  LINKÖPING, \\
		013-28 21 98, matkr@isy.liu.se \\
		\textbf{Kontaktperson hos kund}: Mattias Krysander, 013-28 21 98,matkr@isy.liu.se 
		\\
		\textbf{Kursansvarig}: Tomas Svensson, 3B:528,013 28 21 59,tomass@isy.liu.se
		\\[0.5cm]
		\textbf{Handledare}: Peter Johansson, 013-28 1345 peter.a.johansson@liu.se

				\end{center}
		\endgroup
	\vspace*{\fill}
\newpage

%-----------------------------------------------------------------
%Innehållsföreteckning

\addto\captionsswedish{\renewcommand{\contentsname}{Innehållsförteckning}}

\tableofcontents
\thispagestyle{fancy}
\newpage

\pagenumbering{arabic}
%-----------------------------------------------------------------
%Översikt
\section{Inledning}
\section{Problemformulering}
\section{Principer för trådlös kommunikation}
\subsection{WLAN}
Ett wireless local area network (WLAN) kopplar ihop två eller flera noder med hjälp av någon form av trådlös distributionsmetod. Ett exempel på en sådan distributionsmetod är Orthogonal frequency-division multiplexing (OFDM) som utnyttjar olika bärvågfrekvenser får att koda om signalen. WLAN använder sig oftast av en accesspunkt så att noderna/användarna kan förflytta sig inom räckvidden för denna accesspunkt. Idag används mestadels WLAN som är baserade på standarden IEEE 802.11 som i vardagligt språk brukar kallas Wi-Fi. Standarden är skapat av Institute of Electrical and Electronics Engineers, (IEEE) och använder frekvensband på 2.4, 3.6, 5 och 60Hz.

\subsubsection{Stationer}
Alla komponenter i ett WLAN utgörs av antingen stationer som kan vara noder eller accesspunkter. Alla stationerna har ett nätverkskort, wireless network interface controller (WNIC), som jobbar på samma lager som MAC-lagret (Se rubrik IEEE 802.11). Nätverkskortet använder antenn för att kommunicera med hjälp av radiovågor. En accesspunkt utgörs oftast av en router eller en switch medan en nod kan vara en PC eller mobiltelefon. 

\subsubsection{IEEE 802.11}
IEEE 802.11 familjen består av en serie halv-duplex (kommunikationen kan ske i båda riktningarna men bara en riktning i taget) modulationstekniker som använder sig av samma MAC-protokoll. MAC-protokollet eller MAC-lagret är ett sublager i datalänklagret. Detta lager fungerar som en mellanhand mellan logical link control (LLC) och nätverkets fysiska lager och styr alltså hur nätverksnoderna får åtkomst till det fysiska skiktet (signal och binär överföring).
IEEE 802.11 kräver att nod \it(x) konstant lyssnar för att vara redo om en nod \it{y} möjligen skulle försöka skicka data. 

\subsection{Bluetooth}
Bluetooth är en global standard för trådlöskommunikation på korta avstånd. Bluetooth kan sättas upp utan en tidigare existerande trådlös arkitektur och blir därför väldigt attraktiv för att para ihop enheter. Så som Bluetooth protokollet fungerar är det just parning mellan två enheter är aktuellt. Bluetooth har fler aspekter som gör det särdeles bra för små mobila enheter. 
\begin{itemize}
\item Låg strömförbrukning 
\item Låg kostnad
\item Liten formfaktor
\item Behöver ej fri sikt mellan enheterna
\end{itemize}

\subsubsection{Koppling av enheter}
En parkoppling mellan två Bluetooth-enheter, A och B, fungerar enligt principen att ena enheten, låt oss säga enhet B, måsta vara upptäckbar och måste alltså sända ut en form av identifikation, ofta i form av ett namn. Enhet A söker av vilka enheter den kan detektera och ger ofta en komplett lista till användaren över enheter som den detekterade. Nästa steg är att enhet A begär en koppling till enhet B. När kopplingen lyckas blir enhet A master och enhet B slave. Tvåvägskommunikation är nu möjlig mellan enheterna.

\subsubsection{Säkerhet}
Bluetooth är en förhållandevis säker kommunikationsform, men är självklart inte helt säker. En begränsning i säkerheten är att Bluetooth-enheterna ofta saknar inmatningsmöjligheter och display. Detta leder till att lösenord inte kan fyllas i eller ens genereras på enheterna. I fallet av att båda enheter har display och inmatningsmöjligheter kan en 16 siffror lång PIN-kod genereras som sedan måste fyllas i på den enhet som begär kopplingen. Detta skyddar mot de flesta former av avlyssning.
För enheter som saknar antingen display eller inmatningsmöjligheter finns andra möjliga lösningar. NFC alltså Near-Field Communication, kan användas för parning och då måste enheterna i princip röra varandra i parningsfasen men kan därefter utnyttja fulla räckvidden på Bluetooth. Detta kallas Out-of-Band Pairing.

En annan typ av säkerhet som används är "Just Works", som utnyttjas när enheterna saknar både inmatningsmöjligheter och display. Ett exempel på detta är Bluetooth headsets till mobiltelefoner. Denna säkerhetsmetod kräver inte att användare utför någonting förutom själva kopplingen. Denna metod skyddar mot passiv avlyssning, men ej mot aktiv avlyssning. 
Aktiv avlyssning är när en enhet lägger sig mellan de två enheterna och vidarebefodrar kommunikationen emellan dem och kan vara väldigt svår att upptäcka.
Passiv avlyssning så försöker man endast lyssna av kommunikationen mellan två enheter. 


 
\subsection{Infrared (IR)}
\subsection{ZigBee}
\subsection{Radiostyrning}
\section{Resultat}
\subsection{Diskussion}

Disposition: 
\\
1. Inledning 
\\
2. Problemformulering (frågeställningar som rapporten ska behandla) 
\\
3. Kunskapsbas (litteratur, datablad, dokumentation, etc.) 
\\
4. Fördjupningstext (modeller, beräkningar, analyser och eventuella experiment) 
\\
5. Resultat och slutsatser 





% ----------------------------- Appendix -----------------------------------------

\newpage
\appendix
\pagestyle{empty}
\newgeometry{left=2cm,right=2cm,bottom=2cm,top=2cm}
\section{Appendix A}

\end{document}