\documentclass[a4paper,12pt,fleqn]{article}
\usepackage{fixltx2e}
\usepackage[utf8]{inputenc}
\usepackage{graphicx}
\usepackage{sidecap}
\usepackage{fancyhdr}
\usepackage{amssymb,amsmath}
\usepackage[swedish]{babel}
\usepackage[margin=1.5in]{geometry}
\usepackage{abstract}
\usepackage[parfill]{parskip}
\usepackage{tocloft}
\usepackage{adjustbox}
\usepackage{textcomp}
\usepackage[T1]{fontenc}
\usepackage{listings}
\usepackage{xcolor,colortbl}
\usepackage{hyperref}

%----------------------------------------------------------------
%C-kod formatering

\definecolor{listinggray}{gray}{0.9}
\definecolor{lbcolor}{rgb}{0.9,0.9,0.9}
\lstset{
backgroundcolor=\color{lbcolor},
    tabsize=4,    
%   rulecolor=,
    language=[GNU]C++,
        basicstyle=\scriptsize,
        upquote=true,
        aboveskip={1.5\baselineskip},
        columns=fixed,
        showstringspaces=false,
        extendedchars=false,
        breaklines=true,
        prebreak = \raisebox{0ex}[0ex][0ex]{\ensuremath{\hookleftarrow}},
        frame=single,
        numbers=left,
        showtabs=false,
        showspaces=false,
        showstringspaces=false,
        identifierstyle=\ttfamily,
        keywordstyle=\color[rgb]{0,0,1},
        commentstyle=\color[rgb]{0.026,0.112,0.095},
        stringstyle=\color[rgb]{0.627,0.126,0.941},
        numberstyle=\color[rgb]{0.205, 0.142, 0.73},
%        \lstdefinestyle{C++}{language=C++,style=numbers}’.
}
\lstset{
    backgroundcolor=\color{lbcolor},
    tabsize=4,
  language=C++,
  captionpos=b,
  tabsize=3,
  frame=lines,
  numbers=left,
  numberstyle=\tiny,
  numbersep=5pt,
  breaklines=true,
  showstringspaces=false,
  basicstyle=\footnotesize,
%  identifierstyle=\color{magenta},
  keywordstyle=\color[rgb]{0,0,1},
  commentstyle=\color{Darkgreen},
  stringstyle=\color{red}
  }
  %-----------------------------------------------------------------
  %marginaler

  \renewcommand{\abstractnamefont}{\normalfont\normalsize\bfseries}
  \renewcommand{\abstracttextfont}{\normalfont\small}
  \renewcommand{\headrulewidth}{0pt}
  \renewcommand{\cftsecleader}{\cftdotfill{\cftdotsep}} 
  \setlength{\absleftindent}{0pt}
  \setlength{\absrightindent}{0pt}
  \setlength{\headheight}{15pt}

  \addtolength{\oddsidemargin}{-.5in}
  	\addtolength{\evensidemargin}{-.5in}
  	\addtolength{\textwidth}{1in}


  %-----------------------------------------------------------------
  %header and footer

  \pagestyle{fancy}
  \lhead{
  	\begin{picture}(0,0)
  		\put(5,0){\includegraphics{logotyp.png}}
  	\end{picture}}
	
  \fancyhead[C]{\small{Mapmaster2001}}
  \fancyhead[R]{\small \today}
  \fancyfoot[L]{\small{TSEA56 \\ LIPS Designspecifikation}}
  \fancyfoot[C]{\small{\thepage}}
  \fancyfoot[R]{\small{Projektgrupp 8 \\ Email}}

  %-----------------------------------------------------------------

%-------------------------------------------------------------------
%Första sidan

\begin{document}
	\pagestyle{fancy}
\pagenumbering{roman}
	\vspace*{\fill}
		\begingroup
			\begin{center}
				\huge{\textbf{MapMaster 2001}}
				\\
				\vspace{5pt}
				\normalsize
				Kandidatprojekt Y - Grupp 8 - VT2014
				\\
				Version 1.0
				\end{center}
		\endgroup
	\vspace*{\fill}
	
	\begin{center} %Börjar centrering 
		Status
		\\
		\vspace{3pt} %Whitespace 3 pts
	    \begin{tabular}{| p{3cm} | p{3cm} | p{3cm} |} %tabell, 4 horizontella |, 3 cm emellan dem.
	    \hline %översta horizontella linjen.
	    Granskad & - & \today \\ \hline % & -tecken för att "gå till nästa ruta" 
		Godkänd & - & - \\ \hline % avslutas med \\ och \hline.

	    \end{tabular}
	\end{center}
	\vspace{2cm}
	\newpage
%-----------------------------------------------------------------
%Projektidentitet
	
	\vspace*{\fill}
		\begingroup
			\begin{center}
				\LARGE{\textbf{PROJEKTIDENTITET}}
				\\
				\footnotesize
				Grupp 8, 2014/VT, MapMaster2001
				\\
				Linköpings tekniska högskola, ISY
				\\
				\vspace{1cm}
	  \begin{tabular}{| p{3cm} | p{4.3cm} | p{2.4cm} | p{3.8cm} |}
	    \hline
		\textbf{Namn} & \textbf{Ansvar} & \textbf{Telefon} & \textbf{E-post} \\ \hline
	    Jens Edhammer & Dokumentanvsvarig (DOK) & 076-030 67 80 & jened502@student.liu.se \\ \hline
		Erik Ekelund & Designansvarig (DES) & 073-682 43 06 & eriek984@student.liu.se \\ \hline
		David Habrman &  & 976-017 71 15 & davha227@student.liu.se \\ \hline 
		Tobias Grundström & Testansvarig (TES) & 073-830 44 45 & tobgr602@student.liu.se \\ \hline 
		Hans-Filip Elo &   & 073-385 22 32 & hanel742@student.liu.se \\ \hline 
		Niklas Ericson & Projektledare (PL) & 073-052 27 05 & niker917@student.liu.se \\ \hline
	    \end{tabular}
		
		\vspace{1cm}
		\textbf{E-postlista för hela gruppen:} mapmaster2001@cyd.liu.se
		\\[0.5cm]
		
		\textbf{Kund}: Mattias Krysander, Linköpings Universitet, 581 83  LINKÖPING, \\
		013-28 21 98, matkr@isy.liu.se \\
		\textbf{Kontaktperson hos kund}: Mattias Krysander, 013-28 21 98,matkr@isy.liu.se 
		\\
		\textbf{Kursansvarig}: Tomas Svensson, 3B:528,013 28 21 59,tomass@isy.liu.se
		\\[0.5cm]
		\textbf{Handledare}: Peter Johansson, 013-28 1345 peter.a.johansson@liu.se

				\end{center}
		\endgroup
	\vspace*{\fill}
\newpage

%-----------------------------------------------------------------
%Innehållsföreteckning

\addto\captionsswedish{\renewcommand{\contentsname}{Innehållsförteckning}}

\tableofcontents
\thispagestyle{fancy}
\newpage

\pagenumbering{arabic}
%-----------------------------------------------------------------
%Översikt

\section{Inledning}
Ge en översiktlig beskrivning av produkten och uppdraget gärna kopplat till bilder.
Lyft gärna fram det som ni anser är utmanande/intressant i uppdraget. 
Beskriv kortfattat dispositionen på rapporten.
\section{Problemformulering}
Redogör kort för kravbilden och referera till projektdirektiv och kravspecifikation för att läsa detaljer.
Lägg återigen mest fokus på det som ni ansåg utmanande/intressant. 
\section{Kunskapsbas}
Beskriv kortfattat och referera litteratur, datablad och dokumentation som ni använt er av för att genomföra projektet. 
\section{Genomförande}
Beskriv hur projektet har bedrivits och referera bland annat till LIPS-modellen, systemskiss, projektplan och designspecifikation. Observera att designspecifikationen belyser arbetsprocessen och inte den slutliga produkten så om ni uppdaterat designspecifikationen löpande under projektet så kan ni infoga t ex den version som var aktuell vid BP3.
Teknisk beskrivning
\section{Teknisk beskrivning}
Redovisa det tekniska resultatet och skrivuppgifterna på ett lämpligt sätt. Detaljerade beskrivningar på grindnivå i hårdvaran eller på instruktionsnivå i mjukvaran är i detta sammanhang oftast ointressant. Referera den tekniska dokumentationen och skrivuppgifterna för beskrivning av detaljer. Lyft fram egna kreativa lösningar!
\section{Resultat}
Beskriv kortfattat hur produkten används, hänvisa till användarmanualen.
Beskriv vilken prestanda produkten har och hur det har testats, hänvisa till eventuella testprotokoll.
Kommentera om produkten klarade de kraven som ställdes upp i kravspecifikationen. 
\section{Slutsatser}
Slutsatser
Sammanfatta arbetet. 
Lyft fram det ni är mest nöjda med.
Reflektera över resultatet såväl tekniskt som över genomförandet. Referera efterstudien.
Framtida arbete
Vad skulle ni göra annorlunda om ni skulle göra om samma uppdrag?
Vad skulle ni vilja utveckla om ni fick mer tid?
Hur skulle ni tänka er att ändra uppgiften för att göra den ännu mer intressant?
\section{Referenser}


% ----------------------------- Appendix -----------------------------------------
% --------------------------------------------------------------------------------

\newpage
\appendix
\pagestyle{empty}
\newgeometry{left=2cm,right=2cm,bottom=2cm,top=2cm}
\section{Appendix A}
LIPS-dokument och fördjupningar enligt ovan.
\end{document}